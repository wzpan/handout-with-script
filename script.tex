% !Mode:: "TeX:UTF-8"
% Copyright (C) \the\year by Joseph Pan <cs.wzpan@gmail.com>
% This file may be distributed and/or modified under the
% conditions of the LaTeX Project Public License, either
% version 1.2 of this license or (at your option) any later
% version. The latest version of this license is in:
%
% http://www.latex-project.org/lppl.txt
%
% and version 1.2 or later is part of all distributions of
% LaTeX version 1999/12/01 or later.

\documentclass[10pt]{scrartcl}
\usepackage{handoutWithScript}

\begin{document}

\title{Handout-With-Script}

\author{Joseph Pan}

\maketitle

\begin{myscript}

  Welcome to use \textcolor{blue}{Handout-With-Script}! 

  I'm Weizhou Pan, alternatively you can call me Joseph. $\smiley$

\end{myscript}

\begin{myscript}
  \textsl{Handout-With-Script} is a \LaTeX{} template for generating a handout
  with script for presentation. When you are writing a script, it can
  automatically insert the slide you already created in your slides file into
  your script. This is very handy for someone that need to practice his/her
  presentation with script and the slides at the same time.

  In the following pages we will show how to use \textsl{Handout-With-Script} to
  write a script.
\end{myscript}

\begin{myscript}
  Before we get started, let's find out what are the dependecies of 
  \textsl{Handout-With-Script}: 

  \begin{enumerate}
  \item \href{http://www.pdflabs.com/tools/pdftk-the-pdf-toolkit/}{\textbf{pdftk}} -
    a CLI tool for manipulating PDF documents. We will use it to split the
    slides into pages. PDFtk Server can runs on Windows, Mac OS X and
    Linux, which makes our template cross-platform. \click
  \item \textbf{slides.pdf} - your slides for writing script for. It's not
    strange that you should make slides before writing scripts, right? $\smiley$
  \end{enumerate}
\end{myscript}

\begin{myscript}
  So the pipeline of using \textsl{Handout-With-Script} goes like this: 

  \begin{enumerate}
  \item Make your slides; \click
  \item Output it as pdf file. Almost all the format of slides offers the
    functions to export the slides as pdf format. \textsl{e.g.} Microsoft
    Powerpoint, Libre Office, etc. If you use \LaTeX{}+Beamer to write slides,
    make sure you use the ``handout'' document class option, so as to output as
    a handout without the redundent overlay pages; \click
  \item Rename it as \verb|slides.pdf| and put it into the same folder as
    \verb|script.tex|; \click
  \item Write your script! 
  \end{enumerate}
\end{myscript}

\begin{myscript}
  \textsl{Handout-With-Script} is very simple and there are only 3 usages you
  need to keep them in mind. First of all, it offers 2 \LaTeX{} environments: 

  The first one is \verb|myscriptpage|, which \textcolor{blue}{1)} creates a new
  page where you can write your script, and \textcolor{blue}{2)} inserts a slide at the top of the
  page. The slide comes from a specified page from \verb|slides.pdf|.  When
  using it, a page number is also needed to point out which slide to insert.
  \click

  But most of the time you \textbf{DO NOT} have to explicitly point out the page
  number! An intuitive way to liberate us from typing such stupid numbers is
  that we always play the slides page-by-page, in other words, incrementally! So
  here comes another environment called \verb|myscript|, which inserts a slide
  that implicitly specified by a page counter. When you finished writing the
  script of this page by ending as \verb|\end{mycript}|, the page counter will
automatically be added with 1 so as to point to next page.

Also we offer a command \verb|\click|, which will output a sweet \click icon
that indicates a mouse clicking. It will be useful for some slides that contains
many layouts and animations and may confuses the presenter when to click the
mouse.

  In the following slides we will show some examples of the above two environments.
\end{myscript}

\begin{myscriptpage}{10}
  This page will insert the slide 10.
\begin{verbatim}
    \begin{myscriptpage}{10}
      This page will insert the slide 10.
    \end{myscriptpage}
\end{verbatim}
\end{myscriptpage}

\begin{myscript}
  This page will insert the slide \theslidenum.
\begin{verbatim}
    \begin{myscript}
      This page will insert the slide \theslidenum.
    \end{myscript}
\end{verbatim}
\end{myscript}

\begin{myscript}
  This page will insert the slide \theslidenum.
\begin{verbatim}
    \begin{myscript}
      This page will insert the slide \theslidenum.
    \end{myscript}
\end{verbatim}
\end{myscript}

\begin{myscript}
  This page will insert the slide \theslidenum.
\begin{verbatim}
    \begin{myscript}
      This page will insert the slide \theslidenum.
    \end{myscript}
\end{verbatim}
\end{myscript}

\begin{myscript}
  This page will insert the slide \theslidenum.
\begin{verbatim}
    \begin{myscript}
      This page will insert the slide \theslidenum.
    \end{myscript}
\end{verbatim}
\end{myscript}

\begin{myscript}
  This page will insert the slide \theslidenum.
\begin{verbatim}
    \begin{myscript}
      This page will insert the slide \theslidenum.
    \end{myscript}
\end{verbatim}
\end{myscript}

\begin{myscriptpage}{1}
  This page will insert the slide 1.
\begin{verbatim}
    \begin{myscriptpage}{1}
      This page will insert the slide 1.
    \end{myscriptpage}
\end{verbatim}
\end{myscriptpage}


\begin{myscript}
  Thank you!
\end{myscript}

\end{document}

%%% Local Variables: 
%%% mode: latex
%%% TeX-master: t
%%% End: 
